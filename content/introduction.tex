\chapter{Introducción}
\label{chap:introduction}

En el ámbito digital es trivial hacer copias de información sin límite con modificaciones igualmente interminables. Para que una moneda exista digitalmente y sea ampliamente adoptada, sus usuarios deben tener la certeza de que su oferta es estrictamente limitada. Un receptor de dinero debe poder verificar que no está recibiendo monedas falsificadas o monedas que ya han sido enviadas a otra persona. Por este motivo, para lograr esto sin requerir la colaboración de terceros como una autoridad central, su suministro y su historial completo de transacciones debe ser públicamente verificable.

Podemos utilizar herramientas criptográficas para permitir que los datos registrados en una base de datos de fácil acceso ---la cadena de bloques--- sean prácticamente inmutables e inamovibles, con una legitimidad que no puede ser cuestionada por ninguna de las partes. Las criptomonedas almacenan transacciones en la cadena, que actúa como un registro contable público\footnote{En este contexto, el concepto de \textit{libro contable} solamente significa un registro de todos los eventos de creación e intercambio. Concretamente, se almacena cuánto dinero es transferido en cada evento y a quién.} de todas las operaciones de la moneda. La mayor parte de las criptomonedas almacenan transacciones en texto claro para facilitar el proceso de verificación de transacciones por la comunidad de usuarios.

Claramente, una cadena de bloques abierta desafía cualquier entendimiento básico de la privacidad, ya que virtualmente publica el historial completo de las transacciones de sus usuarios.

Para hacer frente a la falta de privacidad, los usuarios de criptomonedas como Bitcoin pueden ofuscar las transacciones utilizando direcciones intermedias temporales \cite{DBLP:journals/corr/NarayananM17}. Sin embargo, con las herramientas apropiadas, es posible analizar los flujos y, en cierta medida, conectar a los emisores y receptores reales de cada operación \cite{DBLP:journals/corr/ShenTuY15b, DK-police-tracing-btc, Andrew-Cox-Sandia}.

Por contra, la criptomoneda Monero intenta abordar la cuestión de la privacidad mediante el almacenamiento de direcciones de un solo uso para la recepción de fondos en la cadena de bloques y acreditando la dispersión de los fondos en cada transacción usando firmas en anillo. Con estos métodos no hay una forma efectiva de vincular a los receptores o rastrear el origen de los fondos \cite{Monero-intro}.

Además, las cantidades de las transacciones en la cadena de bloques de Monero están ocultas detrás de construcciones criptográficas, lo que hace que los flujos de monedas también sean opacos. El resultado es una criptomoneda con un alto nivel de privacidad.


\section{Objetivos}
\label{sec:goals}

Monero es una criptomoneda de reciente creación \cite{bitmonero-launched, monero-history}, aunque muestra un crecimiento constante en cuanto a popularidad \footnote{\label{marketcap_note}A 21 de febrero de 2019, Monero ocupa la 13ª posición en cuanto a capitalización de mercado, véase \\ \url{https://coinmarketcap.com/}}. 
Desafortunadamente, hay poca documentación comprensible describiendo los mecanismos que utiliza. Y lo que es peor, partes importantes de su marco teórico han sido publicadas en documentos no revisados por pares que están incompletos o que contienen errores. De hecho, para partes significativas del ecosistema de Monero, la única fuente de información fiable es el propio código fuente de la aplicación.

Además, para aquellos que no tienen experiencia en matemáticas, aprender los fundamentos de la criptografía de curva elíptica, que Monero utiliza ampliamente, puede ser una tarea desestructurada e incluso frustrante. \footnote{Anteriormente, ya se ha intentado explicar el funcionamiento de Monero \cite{MRL-0003}. Sin embargo, este intento no terminó por describir la criptografía de curva elíptica que lo hacía posible quedando incompleto y, tras cinco años, desactualizado.}

Pretendemos paliar esta situación introduciendo los conceptos fundamentales necesarios para entender la criptografía de curva elíptica, revisando los algoritmos y esquemas criptográficos y recogiendo información en profundidad sobre el funcionamiento interno de Monero.

Para proporcionar la mejor experiencia a nuestros lectores, nos hemos preocupado de construir una descripción constructiva, paso a paso, de la criptomoneda de Monero.

En la primera edición de este informe hemos centrado nuestra atención en la versión 7 del protocolo Monero, correspondiente a la versión 0.12.x.x de Monero. Todos los mecanismos relacionados con las transacciones aquí descritos pertenecen de hecho a esas versiones. Los esquemas de transacciones que han quedado obsoletos no han sido explorados hasta el momento, incluso si estos son soportados parcialmente para mantener la compatibilidad con versiones anteriores del \textit{software}.


\section{Destinatarios del documento}

Los autores somos conscientes de que muchos lectores se enfrentarán a este documento con poca o ninguna comprensión de matemática discreta, estructuras algebraicas, criptografía y cadenas de bloques. Hemos tratado de ser lo suficientemente minuciosos como para que perfiles de todas las perspectivas puedan aprender Monero sin necesidad de investigación externa.

Hemos, o bien delegado a notas a pie de página, o bien omitido intencionadamente algunos tecnicismos matemáticos, siempre que su explicación chocase con la claridad que esperamos para este documento. También hemos omitido detalles concretos de aplicación que no creíamos que eran esenciales. Nuestro objetivo ha sido presentar el tema a medio camino entre la criptografía matemática y la programación informática, con el objetivo de lograr una claridad conceptual completa.


\section{Orígenes de la criptomoneda Monero}

The cryptocurrency Monero, originally known as BitMonero, was created in April, 2014 as a derivative of the proof-of-concept currency CryptoNote \cite{bitmonero-launched}. Monero means `money' in the language Esperanto, and its plural form is Moneroj (Moe-neh-rowje, similar to Moneros but using the -ge from orange).

CryptoNote is a cryptocurrency devised by various individuals. A landmark whitepaper describing it was published under the pseudonym of Nicolas van Saberhagen in October, 2013 \cite{cryptoNoteWhitePaper}. It offered receiver anonymity through the use of one-time addresses, and sender ambiguity by means of ring signatures.

Since its inception, Monero has further strengthened its privacy aspects by implementing amount hiding, as described by Greg Maxwell (among others) in \cite{Signatures2015BorromeanRS} and integrated into ring signatures based on Shen Noether's recommendations in \cite{MRL-0005}.
  

\section{Estructura de este documento}

As mentioned, our aim is to deliver a self-contained and step-by-step description of the Monero cryptocurrency. This report has been structured to fulfill this objective, leading the reader through all parts of the currency’s inner workings.

In our quest for comprehensiveness, we have chosen to present all the basic elements of cryptography needed to understand the complexities of Monero, and their mathematical antecedents. In Chapter \ref{chap:basicConcepts} we develop essential aspects of elliptic curve cryptography.

Chapter \ref{chapter:ring-signatures} outlines the ring signature algorithms that will be applied to achieve confidential transactions.

In Chapter \ref{chapter:pedersen-commitments} we introduce the cryptographic mechanisms used to conceal amounts.

With all the components in place, we explain the transaction schemes used in Monero in Chapter \ref{chapter:transactions}.

We unfold the Monero blockchain in Chapter \ref{chapter:blockchain}.

While not essential to the operation of Monero, there is a lot of utility in multisignatures that allow multiple people to send and receive money collaboratively. Simple multisignatures (N-of-N and (N-1)-of-N, with naive key aggregation) are currently available in the source code. However, the Monero Research Lab is currently writing a technical paper on the subject, with several improvements and a security proof `in the works'. Readers can find our writeup (subject to change) on multisignatures here: \url{https://github.com/UkoeHB/Monero-RCT-report}, or look forward to its inclusion in future editions. %We devote Chapter \ref{chapter:multisignatures} to describing Monero's current multisignature approach, and to potential future developments in that area. %This is formally called (N-1)-of-N and N-of-N threshold authentication.

Appendices \ref{appendix:RCTTypeFull} and \ref{appendix:RCTTypeSimple} explain the structure of sample transactions from the blockchain. Appendix \ref{appendix:block-content} explains the structure of blocks (including block headers and miner transactions) in Monero's blockchain, while Appendix \ref{appendix:genesis-block} brings our report to a close by explaining the structure of Monero's genesis block. These provide a connection between the theoretical elements described in earlier sections with their real-life implementation.

We\marginnote{¿Verdad que es útil?} use margin notes to indicate where Monero implementation details can be found in the source code.\footnote{These margin notes are accurate for version 0.12.x.x of the Monero software suite, but may gradually become inaccurate as the code base is constantly changing. However, the code is stored in a git repository (\url{https://github.com/monero-project/monero}), so a complete history of changes is available.} There is usually a file path, such as src/ringct/rctSigs.cpp, and a function, such as \(\textrm{genBorromean()}\). Note: `-' indicates split text, such as crypto- note $\rightarrow$ cryptonote.

\section{Aviso sobre conceptos criptográficos}

All signature schemes, applications of elliptic curves, and Monero implementation details should be considered descriptive only. Readers considering serious practical applications (as opposed to a hobbyist's explorations) should consult original sources and technical specifications (which we have cited where possible). Signature schemes need well-vetted security proofs, and Monero implementation details can be found in Monero's source code. In particular, as a common saying goes, `don't roll your own crypto'. Code implementing cryptographic primitives should be well reviewed by experts, and have a long history of solid performance.

\section{Historia de \textit{Monero desde cero}}

La versión inglesa de \textit{Monero desde cero}, \textit{Zero to Monero} es una expansión de la tesis de maestría de Kurt Alonso \textit{Monero - Privacy in the Blockchain} \cite{kurt-original}.