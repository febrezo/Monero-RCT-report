% this file is called up by main.tex
% content in this file will be fed into the main document

% ---------------------------------------------------------------------------

La criptografía. Puede parecer que solamente los matemáticos y los grandes expertos en informática tienen acceso a esta oscura, esotérica, poderosa y elegante disciplina. Sin embargo, muchos tipos de criptografía son tan simples que cualquiera puede aprender sus conceptos fundamentales.
\\ \newline
Muchas personas saben que se utiliza para asegurar las comunicaciones, ya sea cifrando cartas o permitiendo las interacciones privadas digitales. Otra de sus aplicaciones está en las conocidas como criptomonedas. En primer lugar, estas monedas digitales utilizan la criptografía para garantizar que ninguna unidad monetaria puede ser duplicada o creada a voluntad. Con ese objetivo, muchas criptomonedas confían en la tecnología \textit{blockchain} (o cadena de bloques) para generar libros contables públicos y distribuidos que contienen los registros de todas las transacciones de la divisa de modo que estos puedan ser verificados por terceros actores \cite{Nakamoto_bitcoin}.
\\ \newline
A primera vista, puede parecer que es necesario que las transacciones se envíen y almacenen en texto claro para garantizar que puedan ser verificadas. Sin embargo, es posible ocultar la identidad de sus participantes así como las cantidades implicadas utilizando herramientas criptográficas que permiten a un observador comprobar y consensuar cada transacción \cite{cryptoNoteWhitePaper}. Monero es un ejemplo de ello.
\\ \newline
En este trabajo nos esforzaremos para explicar a cualquier persona que tenga conocimientos de álgebra básica y que entienda conceptos de informática como la representación binaria de un número, no solamente cómo funciona Monero en profundidad, sino también cuán útil y hermosa puede llegar a ser la criptografía que lo hace posible.
\\ \newline
Para los lectores más experimentados, Monero es una criptomoneda basada en \textit{blockchain} concebida como un grafo acíclico distribuido unidimensional convencional (DAG) \cite{Nakamoto_bitcoin} en la que las transacciones están basadas en criptografía de curva elíptica utilizando la curva Ed25519 \cite{Bernstein2008}, las entradas son firmadas utilizando firmas MLSAG (Multilayered Linkable Spontaneous Anonymous Group) de tipo Schnorr \cite{MRL-0005} y las cantidades de las salidas (comunicadas a los receptores por medio de ECDH \cite{Diffie-Hellman}) son ocultadas utilizando compromisos de Pedersen \cite{maxwell-ct} y firmas en anillo borromeas de tipo Schnorr \cite{Signatures2015BorromeanRS}. Gran parte de este trabajo está dedicado a explicar estas ideas.